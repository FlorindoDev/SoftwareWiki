In questo fase analizzeremo le varie ridondanze e vedremo come gestirle
\subsubsection{Data ultima modifica}
\textit{data ultima modifica} è un attributo di pagina e potrebbe essere \textit{derivato}. Però essendo un informazione che deve essere reperita ogni volta che apriamo una pagina è conveniente averla subito disponibile e non  derivarla ogni volta.
Infatti se analizziamo la quantità di accessi ci converrà in grande scala non derivare la data di ultimo accesso. altrimenti dovremmo accedere ogni volta ad operazione utente e autore per capire chi è stato l'ultimo a fare una modifica. 

\subsubsection{Generalità\_Autore}
Questo attributo potrebbe essere derivato Notiamo che; le Generalità(Nome è cognome) del autore non cambiano molto spesso, quindi conviene per una questione di accessi creare un attributo apposito.