\textbf{Utente} $\longmapsto$ Abbiamo scelto come chiave primaria \textit{Email}.\newline{}\newline{}
\textbf{Frase} $\longmapsto$ Non abbiamo nessuna chiave candidata però abbiamo una chiave parziale "\textit{posizione}". quest'ultimo insieme alla chiave esterna di \textit{Pagina} possiamo identificare univocamente una frase. questo perché in una pagina ci potrà essere solo una frase con la stessa posizione. La chiave primaria quindi è composta da \textit{chiave esterna} e \textit{posizione}\newline{}\newline{}
\textbf{Pagina} $\longmapsto$ il \textit{titolo} è una chiave candidata però per un questione di facilita di identificazione della pagina nella frase e anche per il link e meglio utilizzare una chiave auto incrementata di nome \textit{ID\_Pagina}\newline{}\newline{}
\textbf{Operazione\_Autore} $\longmapsto$ dipende sia da \textit{utente} che da \textit{frase} quindi essendo che non ci sono chiavi candidate e nemmeno chiave parziali invece di creare una chiave primaria molto grande raggruppando tutti gli attributi creeremo un codice di nome \textit{ID\_operazione} che si auto incrementa. Questo per facilitare l'accesso\newline{}\newline{}
\textbf{Operazione\_Utente} $\longmapsto$ dipende sia da \textit{utente} che da \textit{frase} quindi essendo che non ci sono chiave candidate e nemmeno chiave parziali invece di creare una chiave primaria molto grande raggruppando tutti gli attributi creeremo un codice di nome \textit{ID\_operazione} che si auto incrementa. Questo per facilitare l'accesso