% Definizione di una nuova colonna per gli attributi con testo allineato a sinistra
%tabella associazioni
\newcolumntype{L}{>{\raggedright\arraybackslash}p{5cm}}

\begin{longtable}{|p{2.5cm}|p{6cm}|L|}

\hline
\textbf{Entità} & \textbf{Descrizione} & \textbf{Attributi} \\
\hline

\endfirsthead
\multicolumn{3}{c}{} \\
\hline
\textbf{Entità} & \textbf{Descrizione} & \textbf{Attributi} \\
\hline
\endhead

Utente & 
\textit{Utente} generico del sistema che può 
essere anche un autore di una pagina wiki &
\textbf{Nome}(String): Nome dell'utente \newline
\textbf{Cognome}(String): Cognome dell'utente \newline
\textbf{Email}(string): Email dell'utente \newline
\textbf{Password}(string): Password dell'utente \newline
\textbf{Genere}(Char): genere dell'utente \newline
\textbf{Autore}(Bit): Indica se l'utente è un autore (1) oppure un utente (0)\\

\hline
Pagina & L'entità pagina si riferisce alla \textit{pagina wiki} creata da un \textit{utente-autore} &
\textbf{ID\_Pagina}(Int): Numero intero auto incrementato che identifica univocamente una pagina dalle altre \newline
\textbf{Generalità autore}(String): Stringa composta dalle generalità  dell'autore della pagina \newline
\textbf{Titolo}(String): Indica il titolo della pagina, inoltre il titolo identifica univocamente una pagina dalle altre \newline
\textbf{Data ultima modifica}(Timestamp): Indica la data in cui è stata effettuata l'ultima modifica accettata sulla pagina 
\textbf{Data Creazione}(Timestamp): Indica la data in cui è stata effettuata la creazione della pagina\\

\hline
Frase & Frase è un elemento essenziale della pagina, perché una o più frasi
formano il testo della pagina &
\textbf{Testo}(String): Indica il contenuto testuale della frase \newline
\textbf{Posizione}(Int): Indica la posizione numerica all'interno della pagina \newline
\textbf{link}(Bit): questo attributo specifica se la frase è un collegamento ad un altra pagina se il valore è 1, se invece è 0 allora vuol dire che Frase non è un link\\

\hline
Operazione utente& 
Operazione utente è un entità \textit{Superclasse} che tiene traccia di tutte gli inserimenti e modifiche apportate, alle frasi di una pagina, dagli utenti, che esse siano accettate oppure no dall'autore della pagina &
\textbf{ID\_operazione}(String): Attributo auto incrementato che identifica l'operazione univocamente dalle altre\newline
\textbf{DataA}(Timestamp): Data in cui è stata accattata o rifiutata la modifica o inserimento dall'autore \newline
\textbf{DataR}(Timestamp): Data in cui è stata proposta la modifica o inserimento \newline
\textbf{Testo}(String): Contiene il testo inserito dall'utente \newline
\textbf{Accettata}(Bit): Indica se l'operazione proposta è stata accettata (1), rifiutata (0), il valore è inizializzato a 0 quando la proposta non è stata ancora accettata\newline
\textbf{Visionata}(Bit): Questo attributo indica se l'autore ha visionato (1) l'operazione oppure no (0) \newline
\textbf{Posizione}(Int): Posizione numerica specificata dall'utente al momento dell'inserimento di una nuova frase \newline
\textbf{Modifica}(Bit): Questo attributo indica se l'operazione è una modifica (1) oppure no (0)\newline
\textbf{link}(Bit): Indica se l'operazione è effettuata su un link (1) oppure no (0)\newline
\textbf{link\_pagina}(Int): Se l'operazione è effettuata su un link, questo attributo specifica a quale pagina la frase si riferisce\\

\hline
Operazione autore & 
Operazione autore è un entità \textit{Superclasse} che tiene traccia di tutte gli inserimenti e modifiche apportate, alle frasi di una pagina da parte dell'autore &
\textbf{ID\_operazione}(String): Attributo auto incrementato che identifica l'operazione univocamente dalle altre\newline
\textbf{Data}(Timestamp): Data in cui è stata apportata la modifica o inserimento \newline
\textbf{Testo}(String): Contiene il testo inserito dall'autore \newline
\textbf{Posizione}(Int): Posizione numerica specificata dall'autore al momento dell'inserimento di una nuova frase \newline
\textbf{Modifica}(Bit): Questo attributo indica se l'operazione è una modifica (1) oppure no (0)\newline
\textbf{link}(Bit): Indica se l'operazione è effettuata su un link (1) oppure no (0)\newline
\textbf{link\_pagina}(Int): Se l'operazione è effettuata su un link, questo attributo specifica a quale pagina la frase si riferisce\\

\hline
\end{longtable}