In questa fase analizzeremo il tipo della generalizzazioni e come ristrutturarle

\subsubsection{Operazione utente e operazione autore}
    Operazione utente è una generalizzazione \textit{totale disgiunta} poiché un operazione è per forza al  più una modifica o un inserimento. Per una questione di accessi è molto meglio accorpare i figli nel padre. Questo perché in una situazione con un numero elevato di pagine potremmo avere di conseguenza tante richieste per la visione della storicità, modifiche proposte o proposte di inserimento. avremo in alcuni casi il doppio dei accessi o in altri casi il quadruplo dei accessi in più. Tutto questo può essere risparmiato accorpando i figli nel padre. Come contro avremo solo un \textit{attributo a null}, quest'ultimo è accettabile (attributo posizione potrà essere null) poiché aggiungendo questo valore a null ci risparmia molti accessi.\newline
    Operazione autore si elimina per lo stesso ragionamento e nello stesso metodo.
    
\subsubsection{Link}
    \textit{link} è una generalizzazione \textit{parziale/disgiunta} poiché una frase può non essere un link ma una frase non può essere altro oltre al link.
    Sempre per una questione di accessi conviene accorpare il \textit{padre} nei \textit{figli}. Questo perché ogni volta che carichiamo una wiki dobbiamo controllare anche se ci sono dei link. Accorpando il figlio nel padre avremo meno  accessi (uno in meno per controllare se una frase è un link e una in meno anche per navigare tra il link e la frase) velocizzando la risposta del database alla richiesta di visualizzazione di pagina. A discapito avremo un valore a \textit{null} se la frase non è un link. Anche se in un testo ci sono \textit{più frasi} che \textit{non sono un link} quindi i valori a null saranno \textit{consistenti}, Questo non ci interessa poiché è un male affrontabile  a noi interessa la \textit{velocità} con cui carichiamo una wiki.
    
\subsubsection{Autore}
    Un \textit{utente} è una generalizzazione \textit{parziale/disgiunta} poiché un utente può non essere un autore  ma una utente non può essere oltre che un autore.
    Sempre per una questione di accessi preferiamo inserire il figlio nel padre, questo ci permetterà di dimezzare gli accessi per capire chi è un autore. A discapito avremo un attributo che ci indicherà se un utente è un autore o meno. Ci saranno sicuramente più utenti che autori quindi potenzialmente ci potrebbero essere tanti valori a \textit{False(0)} ma il costo di questo attributo è di solo 1 bit. Otterremo sia meno accessi ma utilizzeremo più memoria rispetto a una diversa ristrutturazione ma la quantità di memoria utilizzata per salvare l'informazione sarà minima poiché l'attributo occupa solo un bit. Quindi se 300 utenti accedono avrò solo 300 accessi. Non avremo 600 accessi come nel caso di una diversa ristrutturazione; trasformazione della generalizzazione in relazioni. La ristrutturazione padre nel figlio a costo di accessi sarebbe stata uguale me non avrebbe avuto  molto senso a livello logico.