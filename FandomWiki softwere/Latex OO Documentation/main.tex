\documentclass{article}
\usepackage{hyperref}
\usepackage{comment}
\usepackage{longtable}
\usepackage{array}
\usepackage{graphicx}
\usepackage{fancyhdr}
\usepackage[italian]{babel}
\usepackage{amsmath}
\usepackage{enumitem}
%\usepackage{geometry}

\usepackage{tikz} %copertina
\usetikzlibrary{shapes.geometric, calc} %copertina
\usepackage{xcolor} %copertina

\usepackage{listings} %sql
\lstset{
    language=SQL, 
    basicstyle=\small\ttfamily,
    numbers=left, 
    numberstyle=\tiny, 
    frame=line,
    keywordstyle=\color{blue},
    commentstyle=\color{red},
    stringstyle=\color{orange},
    identifierstyle=\color{black},
    numberstyle=\tiny\color{black},
    breaklines=true,
    linewidth=1.185\linewidth
}

\hypersetup{
    colorlinks=true,
    linkcolor=black,      % Colore del testo del link
    citecolor=blue,      % Colore delle citazioni nel testo
    urlcolor=blue,       % Colore degli URL
    linkbordercolor=blue, % Colore del bordo del link
    citebordercolor=blue, % Colore del bordo delle citazioni nel testo
    urlbordercolor=blue,  % Colore del bordo degli URL
    pdfborderstyle={/S/U/W 1}, % Stile del bordo del link (sottolineato)
}

\begin{document}

\begin{tikzpicture}[overlay,remember picture]

\definecolor{Dandelion}{RGB}{240, 225, 48} % Esempio di definizione del colore Dandelion
\definecolor{orange}{RGB}{255, 165, 0} % Esempio di definizione del colore orange
\definecolor{RoyalBlue}{RGB}{65, 105, 225} % Esempio di definizione del colore RoyalBlue
\definecolor{Emerald}{RGB}{46, 204, 113} % Definizione del colore Emerald
\definecolor{OrangeRed}{RGB}{255, 69, 0} % Definizione del colore OrangeRed

\global\def\THETITLE{\selectfont \bf Documetazione Data Base}
\global\def\SUBTITLE{Sistema di gestione di una pagina wiki}
\global\def\THEAUTHOR{Raffaele Raia e Florindo Zecconi}% set document author
\global\def\THEUNIVERSITY{Università di Napoli Federico 	II}% set document university

	% Background color
	%\fill[black!2](current page.south west) rectangle (current page.north east);
	\fill[white!2](current page.south west) rectangle (current page.north east);
	
%	\node[opacity=0.1,left color=white] (image) at (-2,-20) {\includegraphics[scale=1.5, angle=-15,origin=c]{logo-simple}};
		
	\node[opacity=0.1,left color=white] (image) at ($(current page.south east)+(-3.3, 2.0)$) {\includegraphics[scale=2.25, angle=4,origin=c]{Copertina/F2logobianco.jpg}};
	
	% Rectangles
	\shade[left color=Dandelion, right color=blue!40, transform canvas ={rotate around ={45:($(current page.north west)+(0,-6)$)}}] ($(current page.north west)+(0,-6)$) rectangle ++(9,1.5);
	
	\shade[left color=lightgray,right color=blue!50, rounded corners=0.75cm,transform canvas ={rotate around ={45:($(current page.north west)+(0.9,-9.5)$)}}]($(current page.north west)+(0.9,-9.5)$) rectangle ++(15,1.5);
	
%	\shade[left color=lightgray, rounded corners=0.3cm, transform canvas ={rotate around ={45:($(current page.north west)+(.5,-10)$)}}] ($(current page.north west)+(1.5,-9.55)$) rectangle ++(7,.6);
	
	\shade[left color=orange!80, right color=blue!60, rounded corners=0.4cm, transform canvas ={rotate around ={45:($(current page.north)+(-1.5,-3)$)}}] ($(current page.north)+(-1.5,-3)$) rectangle ++(9,0.8);
	
	\shade[left color=red!80, right color=blue!80, rounded corners=0.9cm, transform canvas ={rotate around ={45:($(current page.north)+(-3,-8)$)}}] ($(current page.north)+(-3,-8)$) rectangle ++(15,1.8);
	
	\shade[left color=orange, right color=blue, rounded corners=0.9cm, transform canvas ={rotate around ={45:($(current page.north west)+(4,-15.5)$)}}]
	($(current page.north west)+(4,-15.5)$) rectangle ++(30,1.8);
	
	\shade[left color=RoyalBlue, right color=Emerald, rounded corners=0.75cm, transform canvas ={rotate around ={45:($(current page.north west)+(13,-10)$)}}] ($(current page.north west)+(13,-10)$) rectangle ++(15,1.5);
	
	\shade[left color=OrangeRed, right color=RoyalBlue!80, rounded corners=0.3cm, transform canvas ={rotate around ={45:($(current page.north west)+(18,-8)$)}}] ($(current page.north west)+(18,-8)$) rectangle ++(15,0.6);
	
	\shade[left color=OrangeRed, right color=RoyalBlue!80, rounded corners=0.4cm, transform canvas ={rotate around ={45:($(current page.north west)+(19,-5.65)$)}}] ($(current page.north west)+(19,-5.65)$) rectangle ++(15,0.8);
	
	\shade[left color=OrangeRed, right color=RoyalBlue!80, rounded corners=0.6cm, transform canvas ={rotate around ={45:($(current page.north west)+(20,-9)$)}}] ($(current page.north west)+(20,-9)$) rectangle ++(14,1.2);
	
	% Year
%	\draw[ultra thick,gray]($(current page.center)+(5,2)$) -- ++(0,-3cm) 
%	node[midway, left=0.25cm, text width=10cm, align=right, black!75] {
%		
%		{\fontsize{25}{30} \selectfont \bf Sistema di gestione di una pagina wiki \\[10pt]}
%	}
%	node[midway, right=0.25cm, text width=6cm, align=left, orange] {
%
%		{\fontsize{72}{86.4} \selectfont \coverdate\today}
%	};
	
	% Title
	\node[align=center] at ($(current page.center)+(0,-5)$) {
		{\includegraphics[scale=0.1]{Copertina/pixlr-image-generator-b77a4501-372c-489b-8045-1161501d516d-PhotoRoom.png-PhotoRoom.png}}\\[1cm]
  
		{\fontsize{40}{72} \selectfont {\THETITLE}} \\[1cm]
        {\fontsize{35}{72} \selectfont {\SUBTITLE}} \\[1cm]
		{\fontsize{16}{19.2} \selectfont \textcolor{orange}{ \bf \THEAUTHOR}}\\[3pt]
		\THEUNIVERSITY\\[3pt]
        \today
	};

\end{tikzpicture}

\newpage

\tableofcontents

\newpage

\pagestyle{fancy}
\fancyhead[L]{ }

\section{Class Diagram del Dominio del problema}
\includegraphics[width=1\textwidth]{DominioProbelema.pdf}
\section{Class Diagram del dominio della Soluzione}
\includegraphics[width=1.3\textwidth]{DominioSoluzione.pdf}
\section{Sequance Diagram Schermata Login GUI}
\includegraphics[width=1.3\textwidth]{SequanceLogin.pdf}
\section{Sequance Diagram Funzione getPagina in Controller}
\includegraphics[width=1.3\textwidth]{SequancecreaPagina.pdf}

\end{document}